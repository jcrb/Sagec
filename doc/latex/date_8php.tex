\hypertarget{date_8php}{
\section{R\'{e}f\'{e}rence du fichier date.php}
\label{date_8php}\index{date.php@{date.php}}
}
\subsection*{Fonctions}
\begin{CompactItemize}
\item 
\hyperlink{date_8php_a3}{f\-Date2unix} (\$d)
\item 
\hyperlink{date_8php_a4}{f\-Datetime2unix} (\$d)
\item 
\hyperlink{date_8php_a5}{u\-Date2French} (\$u)
\item 
\hyperlink{date_8php_a6}{u\-Date2Frenchtime} (\$u)
\item 
\hyperlink{date_8php_a7}{u\-Date2Frenchdatetime} (\$u)
\item 
\hyperlink{date_8php_a8}{u\-Datetime2French} (\$u)
\item 
\hyperlink{date_8php_a9}{round\_\-u\_\-date} (\$u)
\end{CompactItemize}
\subsection*{Variables}
\begin{CompactItemize}
\item 
\hyperlink{date_8php_a0}{\$un\_\-jour} = 60$\ast$60$\ast$24
\item 
\hyperlink{date_8php_a1}{\$sept\_\-jour} = 7$\ast$\$un\_\-jour
\item 
\hyperlink{date_8php_a2}{\$semaine} = array(dimanche','lundi','mardi','mercredi','jeudi','vendredi','samedi)
\end{CompactItemize}


\subsection{Documentation des fonctions}
\hypertarget{date_8php_a3}{
\index{date.php@{date.php}!fDate2unix@{fDate2unix}}
\index{fDate2unix@{fDate2unix}!date.php@{date.php}}
\subsubsection[fDate2unix]{\setlength{\rightskip}{0pt plus 5cm}f\-Date2unix (\$ {\em d})}}
\label{date_8php_a3}


Transforme une date au format fran\c{c}ais jj/mm/aaaa en timestamp unix \hypertarget{date_8php_a4}{
\index{date.php@{date.php}!fDatetime2unix@{fDatetime2unix}}
\index{fDatetime2unix@{fDatetime2unix}!date.php@{date.php}}
\subsubsection[fDatetime2unix]{\setlength{\rightskip}{0pt plus 5cm}f\-Datetime2unix (\$ {\em d})}}
\label{date_8php_a4}


Transforme une date au format fran\c{c}ais jj/mm/aaaa hh:mm:ss en timestamp unix \hypertarget{date_8php_a9}{
\index{date.php@{date.php}!round_u_date@{round\_\-u\_\-date}}
\index{round_u_date@{round\_\-u\_\-date}!date.php@{date.php}}
\subsubsection[round\_\-u\_\-date]{\setlength{\rightskip}{0pt plus 5cm}round\_\-u\_\-date (\$ {\em u})}}
\label{date_8php_a9}


Arrondi une date unix en supprimant la partie horaire \hypertarget{date_8php_a5}{
\index{date.php@{date.php}!uDate2French@{uDate2French}}
\index{uDate2French@{uDate2French}!date.php@{date.php}}
\subsubsection[uDate2French]{\setlength{\rightskip}{0pt plus 5cm}u\-Date2French (\$ {\em u})}}
\label{date_8php_a5}


Transforme un timestamp unix en date fran\c{c}aise \hypertarget{date_8php_a7}{
\index{date.php@{date.php}!uDate2Frenchdatetime@{uDate2Frenchdatetime}}
\index{uDate2Frenchdatetime@{uDate2Frenchdatetime}!date.php@{date.php}}
\subsubsection[uDate2Frenchdatetime]{\setlength{\rightskip}{0pt plus 5cm}u\-Date2Frenchdatetime (\$ {\em u})}}
\label{date_8php_a7}


Transforme un timestamp unix en date:heure fran\c{c}aise \hypertarget{date_8php_a6}{
\index{date.php@{date.php}!uDate2Frenchtime@{uDate2Frenchtime}}
\index{uDate2Frenchtime@{uDate2Frenchtime}!date.php@{date.php}}
\subsubsection[uDate2Frenchtime]{\setlength{\rightskip}{0pt plus 5cm}u\-Date2Frenchtime (\$ {\em u})}}
\label{date_8php_a6}


Transforme un timestamp unix en heure fran\c{c}aise \hypertarget{date_8php_a8}{
\index{date.php@{date.php}!uDatetime2French@{uDatetime2French}}
\index{uDatetime2French@{uDatetime2French}!date.php@{date.php}}
\subsubsection[uDatetime2French]{\setlength{\rightskip}{0pt plus 5cm}u\-Datetime2French (\$ {\em u})}}
\label{date_8php_a8}


Transforme un timestamp unix en date fran\c{c}aise 

\subsection{Documentation des variables}
\hypertarget{date_8php_a2}{
\index{date.php@{date.php}!$semaine@{\$semaine}}
\index{$semaine@{\$semaine}!date.php@{date.php}}
\subsubsection[\$semaine]{\setlength{\rightskip}{0pt plus 5cm}\$semaine = array(dimanche','lundi','mardi','mercredi','jeudi','vendredi','samedi)}}
\label{date_8php_a2}


\hypertarget{date_8php_a1}{
\index{date.php@{date.php}!$sept_jour@{\$sept\_\-jour}}
\index{$sept_jour@{\$sept\_\-jour}!date.php@{date.php}}
\subsubsection[\$sept\_\-jour]{\setlength{\rightskip}{0pt plus 5cm}\$sept\_\-jour = 7$\ast$\$un\_\-jour}}
\label{date_8php_a1}


\hypertarget{date_8php_a0}{
\index{date.php@{date.php}!$un_jour@{\$un\_\-jour}}
\index{$un_jour@{\$un\_\-jour}!date.php@{date.php}}
\subsubsection[\$un\_\-jour]{\setlength{\rightskip}{0pt plus 5cm}\$un\_\-jour = 60$\ast$60$\ast$24}}
\label{date_8php_a0}


Constantes 